\section{Getting Started}
\label{section:cg}

Wir beginnen mit dem Schnelleinstieg. 

\subsection{Voraussetzungen}

Zun�chst m�ssen auf dem Rechner die Voraussetzungen sichergestellt werden:
\begin{itemize}
\item Oracle Java 1.8 oder neuer \\ (\emph{http://www.oracle.com/technetwork/java/javase/overview/index.html})
\item Gradle 2.6 oder neuer (\emph{https://gradle.org/})
\item Git 2.3 oder neuer (unterschiedliche Quellen je nach OS)
\item Entwicklungsumgebubng
	\begin{itemize} 
	\item (entweder) Eclipse 4.5 (Mars) oder neuer (\emph{https://eclipse.org/})
	\item (oder) Intellij IDEA 14.X ((\emph{https://www.jetbrains.com/idea/})
	\end{itemize}
\end{itemize}

\subsection{Repository Auschecken [nicht notwendig bei Verwendung von IntelliJ]}

Das Repository mit allen notwendigen Sourcen und sonstigen Dateien finden sich unter folgender URL: \\
\emph{git.informatik.haw-hamburg.de/srv/git/computergrafik/computergraphics}.\\

\begin{itemize}
	\item Es muss auf den Entwicklungsrechner gecloned werden:\\
	\verb+git clone ssh://<login>@git.informatik.haw-hamburg.de/+\\
	\verb+     srv/git/computergrafik/computergraphics+
	\item \textbf{Kompilieren mit Gradle:} Gradle bietet die M�glichkeit, das Projekt �ber die Kommandozeile zu kompilieren. Dazu muss im Projektverzeichnis (Elternverzeichnis der Unterprojekte wie \emph{graphics\_core}) folgender Befehl ausgef�hrt werden:\\
	\verb+gradle build+
\end{itemize}

\subsection{Entwicklungsumgebung}

Im Prinzip kann man mit jeder beliebigen Entwicklungsumgebung (oder einem Text-Editor) arbeiten. Anleitungen werden hier f�r Eclipse und IntelliJ gegeben.

\subsubsection{Eclipse}

\begin{itemize}
	\item \textbf{Projekte erstellen:} Zum Erstellen der Eclipse-Projektdateien wird wieder Gradle verwendet:\\
	\verb+gradle eclipse+\\
	Nach erfolgreicher Ausf�hrung dieses Befehls befinden sich Eclipse-Projektdateien (\emph{.project}, ...) in jedem der Unterprojekte.
	\item \textbf{Import:} Je nachdem, was genau gemacht werden soll, ben�tigt man verschiedene Unterprojekte. Beispielhaft soll hier die Demo-Anwendung \verb+ObjTriangleMesh+ ausgef�hrt werden. Diese befinden sich in dem Unterprojekt \emph{apps}. Um das Projekt in Eclipse zu importieren sind die folgenden Schritte notwendig:
	\begin{itemize}
		\item File
		\item Import
		\item Existing Projects into Workspace
		\item Select root directory: $<$\emph{Hier muss das Unterverzeichnis \emph{apps} ausgew�hlt werden}$>$
		\item Finish
	\end{itemize}
	Nun ist das Projekt in Eclipse importiert und erscheint im Projekt-Exporer. 
\end{itemize}

\subsubsection{IntelliJ}

\begin{itemize}
\item \textbf{Intellij konfigurieren}, so dass es den nativen SSH-Client benutzt (openSSH unter Linux Mint). Der schon in Intellij eingebaute SSH-Client hat mit dem Git-Server der HAW bei mir nicht funktioniert.)
Configure $\rightarrow$ Settings $\rightarrow$ Version Control $\rightarrow$ Git $\rightarrow$ SSH executable build-in auf native �ndern.
\item \textbf{Auschecken des Repositories:} Check out from Version Control $\rightarrow$ Git\\
Git Repository URL:\\
\verb+ ssh://<login>@git.informatik.haw-hamburg.de/+\\
\verb+     srv/git/computergrafik/computergraphics+\\
Parent Directory: Standardgem��:\\ 
\verb+/home/<account>/IdeaProjects+\\
oder\\ 
\verb+C:\Users\<account>\IdeaProjects+\\
Directory Name: \verb+computergraphics+. Anschlie�end clonen.
\item \textbf{Importieren des Projektes mit Gradle:} Nach dem clonen fragt Intellij, ob die \emph{gradle.build} Datei ge�ffnet werden soll. Dies best�tigen wir mit Ja. Ist Gradle auf dem Rechner installiert muss man hier lediglich den Pfad angeben. Ansonsten bietet Intellij die Option einen \emph{Customizable Gradle Wrapper} zu benutzen. Im zweiten Fall ist die Installation von Gradle nicht notwendig.

\end{itemize}

\subsection{Abschluss}

\begin{itemize}
	\item \textbf{Asset-Pfad einrichten:} Viele Programme verwenden Daten (z.B. Icons oder Dreiecksnetze). Standarddaten finden sich im Projektverzeichnis im Unterverzeichnis \emph{assets}. Damit ausgef�hrte Programme diese Assets auch finden, muss der Pfad zu diesem Verzeichnis bekannt gegeben werde. Dazu wird der absolute Pfad des Unterverzeichnisses \emph{assets} in der Textdatei \emph{resources.txt} eingetragen.
	\item \textbf{Demo-Programm starten:} Im Package \verb+cgresearch.apps.trianglemeshes+ kann nun die Klasse \verb+ObjTriangleMesh+ durch \emph{Rechtsklick} $\rightarrow$ \emph{Run As} $\rightarrow$ \emph{Java Application} ausgef�hrt werden. Hat alles geklappt, dann �ffnet sich ein Fenster und ein 3D-Modell erscheint.
\end{itemize}

%\subsubsection{libGDX}
%
%Zun�chst muss die Entwicklungsumgebung f�r die Verwendung von libGDX angepasst werden. In dieser Anleitung wird zun�chst davon ausgegangen, dass die Anwendung f�r Desktop (egal welches Betriebssystem) und f�r Android Mobil verwendet werden soll. Au�erdem wird davon ausgegangen, dass Eclipse als Entwicklungsumgebung verwendet wird. Zum Einrichten folgt man am Besten der Anleitung zur Einrichtung von libGDX\footnote{http://libgdx.badlogicgames.com/documentation.html}. Prinzipiell m�ssen folgende Komponenten eingerichtet werden:
%\begin{itemize}
%\item JDK 7+
%\item Android SDK
%\item Eclipse Plugin zur Android-Entwicklung
%\item Eclipse Plugin zur Verwendung von Grade (wird f�r das Setup von libGDX ben�tigt)
%\end{itemize} 
%
%Dann wird das zugeh�rige Repository gecloned. Das Repository findet sich unter folgender URL: \emph{git.informatik.haw-hamburg.de/srv/git/computergrafik/cg\_libgdx}.
%
%Schliesslich wird das Projekt in Eclipse importiert. Dazu w�hlt man
%\begin{itemize}
%\item File
%\item Import
%\item Gradle
%\item Gradle Project
%\item Verzeichnis mit dem Repository ausw�hlen
%\item Build Model
%\end{itemize}
%
%In Eclipse werden damit mehrere neue Projekte eingerichtet, wir betrachten hier zun�chst nur 
%\begin{itemize}
%\item \textbf{cg\_libgdx-core:} Hier ist die Anbindung zwischen dem cgresearch Framework und libGDX umgesetzt. Dieses Package muss nur ver�ndert werden, wenn die Anbindung erweitert werden soll, z.B. wenn man eine neue RenderObjectFactory f�r libGDX hinzuf�gen will.
%\item \textbf{cg\_libgdx-desktop:} Dies ist das Startprojekt f�r Desktop-Anwendungen. Um eine Anwendung mit libGDX zu starten, muss man diese in der Klasse \verb+DesktopLauncher+ in den Konstruktor von LibGdxFrame als Parameter stecken. Dann kann man die Anwendung mit libGdx-Rendering auf dem Desktop starten.
%\item \textbf{cg\_libgdx-android:} Dies ist das Startprojekt f�r Android-Anwendungen. Um eine Anwendung mit libGDX zu starten, muss man diese in der Klasse \verb+AndroidLauncher+ in den Konstruktor von LibGdxFrame als Parameter stecken. Dann kann man die Anwendung mit libGdx-Rendering auf dem Desktop starten.
%\end{itemize}
%
%Damit die libGDX-Projekte die Funktionalit�t aus dem cgresearch Framework kennen, muss dieses in Form eines .jars inkludiert werden. Dazu geht man folgendermassen vor:
%
%\begin{itemize}
%\item \textbf{Erzeugen des Jars:} Um aus dem cgresearch Framework ein Har-Archiv zu erzeugen starten man in der Kommandozeile nacheinander die folgenden zwei Befehlen (setzen Maven voraus!):
%\begin{itemize}
%\item \verb+mvn clean+
%\item \verb+mvn package+
%\end{itemize}
%Das erzeugte Jar mit dem Namen cgreasearch.jar findet man dann im Unterverzeichnis \emph{target}
%\item \textbf{Jar in das libGdx-Projekt bewegen:} Die Jar-Datei muss an in das libGdx Projekt bewegt werden. Es soll in folgedem Verzeichnis liegen: \emph{core/build/libs}.
%\item \textbf{Bibliothek in Projekte einf�gen} Zuletzt muss das Jar als externe Bibliothek in alle drei libGdx-Projekte integriert werden. Das geschieht �ber \emph{Rechtsklick auf das Projekt} $\rightarrow$ \emph{Properties} $\rightarrow$ \emph{Java Build Path} $\rightarrow$  \emph{Libraries} $\rightarrow$  \emph{Add External Jar}. 
%\end{itemize}
%
%\textbf{Resourcen/Assets} Ressourcen und Assets, die unter libGdx verwendet werden sollen m�ssen auch im \emph{assets}-Ordner libGdx-Projekt liegen. Der \emph{assets}-Ordner liegt im Unterordner \emph{android}. F�r den Zugriff aus dem Desktop-Projekt muss der Ordner entweder kopiert werten oder es muss ein Symbolischer Link (\emph{ln -s}) erzeugt werden. Au�erdem muss man in der Resourcen-Konfigurationsdatei \emph{resources.ini} den Pfad je nach Betriebssystem anpassen (siehe Abschnitt \ref{section:ressourcen}).
%