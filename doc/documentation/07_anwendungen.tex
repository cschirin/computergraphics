\section{Anwendungen}

In diesem Kapitel werden spezialisierte Dokumentationen zu einzelnen Anwendungen gesammelt.

\subsection{Serielles Arduino Protokoll}

Alle Befehle haben die folgende Form:\\
return-string $\leftarrow$ command-string\\

Im Folgenden werden die Kommandos in einer Grammatik-�hnlichen Darstellung beschrieben. Befehle in Anf�hrungszeichen sind Strings, Befehle ohne Anf�hrungszeichen sind Platzhalter.

command-string = gyro-command $|$ servo-command $|$ distance-command\\
gyro-command = 'gyro'\\
servo-command = 'servo' servo-subcommand\\
servo-subcommand = servo-attach $|$ servo-detach $|$ servo-degree\\
servo-attach = 'attach' pin (= Servo mit dem angegebenen Pin verbinden)\\
servo-detach = 'detach' pin (= Servo an dem angegebenen Pin trennen)\\
servo-degree = 'degree' pin degree (= Servo an dem angegebenen Pin )\\
pin (Pin-Index am Arduino, Ganzzahl)\\
degree (Drehwinkel des Servos, Ganzzahl)\\
distance-command = 'distance' trigger-pin echo-pin\\
echo-pin (Pin-Index am Arduino, Ganzzahl)\\
trigger-pin (Pin-Index am Arduino, Ganzzahl)\\

\subsubsection{Gyro}

Beispiel: 'gyro'\\ (Fragt die Werte aus dem Gyro ab)\\
Liefert einen String aus 6 Fliesskommazahlen, die durch Leerzeichen voneinander getrennt sind. Die ersten drei Zahlen repr�sentieren die drei Koordinaten der Beschleunigung und die weiteren drei Zahlen sind die drei Koordinaten des Gyros.

\subsubsection{Servo}

Beispiel: 'servo degree 2 45'\\ (Setzt den Servo auf Pin 2 auf 45 Grad)\\
Liefert keine R�ckgabe

\subsubsection{Distance (HC-SR04}
Beispiel: 'distance 5 4'\\ (Fragt aktuellen Entfernungswert ab, verwendet Pin 5 f�r den Trigger und Pin 4 f�r das Echo)\\
Gemessene Entfernung in cm.